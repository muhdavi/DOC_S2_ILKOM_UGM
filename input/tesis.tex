%%%%%%%%%%%%%%%%%%%%%%% Kepada prodi s2 ilkom
\newgeometry{top=0.3cm,bottom=0.1cm,left=2cm,right=2cm}
\newpage
\noindent
Kepada: \\
Yth. Pengelola Program Studi \@program \\
Program Pascasarjana FMIPA UGM \\ [.2cm]
Dengan hormat, \\
Yang bertanda tangan dibawah ini \\ [-.2cm]

\indent
\begin{tabular}{p{5cm}p{0.01cm}p{11cm}}
Nama 						& : & \@fullname \\
No. Mahasiswa				& : & \@idnum \\
Program Studi				& : & \@program \\
Biaya Studi/Sumber Beasiswa & : & Biaya Sendiri \\ [.2cm]
\multicolumn{3}{l}{\textbf{Asal S1 (sesuai pada ijazah/transkrip)}} \\
Program Studi	& : & \@prodiasal \\		
Jurusan			& : & \@jurusanasal \\
Fakultas		& : & \@facultyasal \\
Universitas		& : & \@universityasal, IPK: 3,81, Th. Lulus: 2011 \\
\end{tabular}
dengan ini mengajukan permohonan ujian \soutthick{Pra-Tesis/}Tesis saya yang berjudul:

\vspace{.2cm}
\indent
\begin{tabular}{p{15cm}}
\@titleind \\ 
(\@titleeng)
\end{tabular}

\vspace{.2cm}
\noindent
dengan Pembimbing: \@firstsupervisor 

\vspace{.2cm}
\noindent
Sebagai kelengkapan administrasi bersama ini saya lampirkan:
\begin{enumerate}
	\item[\textbf{I.}] \textbf{SYARAT UJIAN PRA-TESIS}
	\begin{enumerate}
		\setlength\itemsep{0.1mm}
		\item Tesis rangkap 4/5 exp.
		\item Komentar dan saran penguji Proposal 4/5 exp.
		\item Surat Keterangan Tesis siap diajukan.
		\item \textit{Check list} isi naskah yang sudah ditandatangani.
		\item \textit{Check list} format Penulisan yang sudah ditandatangani.
		\item Fotocopy KRS Terakhir.
		\item Fotocopy Sertifikat TOEFL dan TPA *).
		\item Surat Pernyataan Publikasi dan Surat Pernyataan Tesis.
		\item Surat Pernyataan Pembuatan Program.
	\end{enumerate}
	\item[\textbf{\circled{II.}}] \textbf{SYARAT UJIAN TESIS AKHIR}
	\begin{enumerate}
		\setlength\itemsep{0.1mm}
		\item Tesis rangkap 4/5 exp.
		\item Draf Naskah Publikasi 4/5 exp.
		\item Komentar dan saran penguji Pra-Tesis 4/5 exp.
		\item Surat Keterangan Tesis siap diajukan.
		\item \textit{Check list} isi naskah yang sudah ditandatangani.
		\item \textit{Check list} format penulisan yang sudah ditandatangani.
		\item Fotocopy Bukti Penyerahan Publikasi akan diterbitkan/review dari tempat yang disetujui.
		\item Kartu Bimbingan yang telah disahkan oleh Kaprodi dan sudah melakukan bimbingan minimal 7 kali (tertulis dalam kartu dan ditandatangani pembimbing).
		\item Fotocopy KRS Terakhir.
		\item Fotocopy Sertifikat TOEFL dan TPA *).
		\item Keterangan Kemajuan Belajar atau Rekap Nilai dari palawa dengan  IP > 2,75.
		\item Fotocopy Bukti Pembayaran SPP Terakhir.
		\item Fotocopy Kartu Mahasiswa Terbaru.
		\item Fotocopy Transkrip Nilai \& Ijazah S1 berbahasa Indonesia maupun Inggris.
	\end{enumerate}
\end{enumerate}

\noindent
Demikian  surat permohonan ini, atas bantuannya saya ucapkan terima kasih.

\vspace{0.2cm}
\noindent
\begin{tabular}{p{10cm}p{10cm}}
	& \@city,\space\today \\
	& Pemohon, \\ [1.1cm]
	& \@fullname
\end{tabular}

\begin{footnotesize}
\begin{tabbing}
Ket\=erangan:  \\
*) \> Skor TOEFL $\geq$ 450, AcEPT $\geq$ 209, TPA/PAPs $\geq$ 500
\end{tabbing}
\end{footnotesize}
\restoregeometry
%%%%%%%%%%%%%%%%%%%%%%%


%%%%%%%%%%%%%%%%%%%%% Surat Keterangan Tesis siap diajukan
\newpage
\begin{center}
{\normalfont\large\bfseries\expandafter{SURAT KETERANGAN}}\par\nobreak
\end{center}

\vspace{1.0cm}
\noindent
Dengan ini kami menerangkan bahwa mahasiswa tersebut di bawah ini:

\vspace{.2cm}
\begin{tabular}{p{2.5cm}p{0.01cm}p{9cm}}
Nama 			& : &\@fullname \\
No. Mahasiswa	& : &\@idnum \\
Program Studi	& : &\@program \\			
Departemen		& : &\@dept
\end{tabular}

\vspace{0.2cm}
\noindent
telah selesai menyusun tesis dengan judul:

\vspace{0.2cm}
\indent
\begin{tabular}{p{14cm}}
\@titleind \\ 
(\@titleeng)
\end{tabular}

\vspace{0.1cm}
\noindent
dan siap untuk diujikan. \\

\noindent
Demikian surat keterangan ini dibuat untuk dapat dipergunakan seperlunya.

\vspace{1cm}
\noindent
\begin{tabular}{p{7cm}p{7cm}}
	& \@city,\space\today \\
	& Pembimbing, \\ [1.5cm]
	& \underline{\@firstsupervisor} \\
	& \@firstsupervisornip
\end{tabular}
%%%%%%%%%%%%%%%%%%%%%%%

%%%%%%%%%%%%%%%%%%%%%%% Form check list format naskah
\newgeometry{top=2.5cm,bottom=.7cm}
\newpage
\begin{center}
{\normalfont\large\bfseries\expandafter{Form check list isi naskah}}
\par\nobreak
\end{center}

\normalsize
\begin{tabbing}
	Nama Mahasiswa \= : \@fullname \\
	Pembimbing \> : \@firstsupervisor
\end{tabbing}

\noindent
Dengan ini saya menyatakan telah memenuhi aturan isi naskah tesis seperti ditunjukkan pada point-point yang saya beri centang $\surd$ berikut ini:

\noindent
\begin{tabular}{|c|m{13cm}|p{.5cm}|}
	\hline
	\multicolumn{2}{|l|}{\head{PENDAHULUAN}} & \multicolumn{1}{c|}{} \\
	\hline
	\multicolumn{2}{|l|}{Latar belakang} & \multicolumn{1}{c|}{} \\
	\hline
  	\colnumber & Menjelaskan topik penelitian dan didukung dengan berbagai data kuantitatif maupun kualitatif yang relevan. & $\surd$ \\ 
  	\hline
  	\colnumber & Mengidentifikasi beberapa faktor/aspek yang terkait permasalahan penelitian. & $\surd$ \\
  	\hline
	\multicolumn{2}{|l|}{\head{TINJAUAN PUSTAKA}} & \multicolumn{1}{c|}{} \\
	\hline
  	\colnumber & Mengkaji tentang beberapa penelitian sejenis yang dilakukan sebelumnya. & $\surd$ \\
  	\hline
  	\colnumber & Menegaskan titik perbedaan penelitian dengan beberapa penelitian sebelumnya. & $\surd$ \\
  	\hline
  	\colnumber & Tabel perbedaan penelitian sejenis dan penelitian yang dilakukan. & $\surd$ \\
  	\hline
	\multicolumn{2}{|l|}{\head{LANDASAN TEORI}} & \multicolumn{1}{c|}{} \\
	\hline
  	\colnumber & Menjelaskan core teori yang tepat dengan masalah penelitian dan dielaborasi secara lengkap. & $\surd$ \\
  	\hline
	\multicolumn{2}{|l|}{\head{RANCANGAN SISTEM}} & \multicolumn{1}{c|}{} \\
	\hline
  	\colnumber & Memberi deskripsi sistem yang dibuat. & $\surd$ \\
  	\hline
  	\colnumber & Menggambarkan Arsitektur Sistem/komponen sistem. & $\surd$ \\
  	\hline
  	\colnumber & Menerapkan metode yang dipakai untuk menyelesaikan masalah penelitian. & $\surd$ \\
  	\hline
  	\colnumber & Mengungkapkan rancangan sesuai dengan model yang dipilih (sebaiknya memakai tools). & $\surd$ \\
  	\hline
  	\colnumber & Menjelaskan perhitungan secara manual. & $\surd$ \\
  	\hline
	\multicolumn{2}{|l|}{\head{IMPLEMENTASI}} & \multicolumn{1}{c|}{} \\
	\hline
  	\colnumber & Menjelaskan implementasi modul-modul berdasar rancangannya. & $\surd$ \\
  	\hline
	\multicolumn{2}{|l|}{\head{HASIL dan PEMBAHASAN}} & \multicolumn{1}{c|}{} \\
	\hline
  	\colnumber & Menguji sistem dengan data riil/data realistik (memenuhi karakteristik data riil) dan membahasnya secara detil. & $\surd$ \\
  	\hline
  	\colnumber & Membandingkan secara kuantitatif dengan hasil penelitian yang lain. & $\surd$ \\
  	\hline
	\multicolumn{2}{|l|}{\head{KESIMPULAN dan SARAN}} & \multicolumn{1}{c|}{} \\
	\hline
	\multicolumn{2}{|l|}{Kesimpulan} & \multicolumn{1}{c|}{} \\
	\hline
  	\colnumber & Mengungkapkan kemampuan dan kekurangan sistem didasarkan dari hasil penelitian yang telah dibahas. & $\surd$ \\
  	\hline
  	\colnumber & Bukan hal-hal yang bersifat umum, atau hal-hal yang didasarkan pada asumsi penulis. & $\surd$ \\
  	\hline
	\multicolumn{2}{|l|}{Saran: (future works)} & \multicolumn{1}{c|}{} \\
	\hline
  	\colnumber & Penelitian lanjut dari penelitian ini untuk perbaikan sistem. & $\surd$ \\
  	\hline
  	\colnumber & Perbaikan dari kekurangan penelitian. & $\surd$ \\
  	\hline
	\multicolumn{2}{|l|}{\head{NASKAH PUBLIKASI (khusus untuk mahasiswa yang ujian tesis)}} & \multicolumn{1}{c|}{} \\
	\hline
  	\colnumber & Sudah ditulis sesuai dengan format IJCCS atau IJEIS. & $\surd$ \\
  	\hline  	
\end{tabular}

\vspace{.5cm}
\noindent
\begin{tabular}{p{10cm}p{10cm}}
						& \@city,\space\today \\
Mengetahui pembimbing,	& Yang membuat pernyataan \\ [1.5cm]
\underline{\@firstsupervisor}	& \underline{\@fullname\space\space\space\space\space\space\space\space\space\space\space\space\space\space\space\space\space\space} \\
\@firstsupervisornip			& NIM. \@idnum

\end{tabular}
%%%%%%%%%%%%%%%%%%%%%%%

%%%%%%%%%%%%%%%%%%%%%%% Form check list format penulisan
\newpage
\begin{center}
{\normalfont\large\bfseries\expandafter{Form check list format penulisan}}
\par\nobreak
\end{center}

\normalsize
\begin{tabbing}
Nama Mahasiswa \= : \@fullname \\
Pembimbing \> : \@firstsupervisor
\end{tabbing}

\noindent
Dengan ini saya menyatakan telah memenuhi aturan format penulisan seperti ditunjukkan pada point-point yang saya beri centang $\surd$ berikut ini:

\noindent
\begin{tabular}{|c|m{14cm}|p{.5cm}|}
	\hline
	\multicolumn{2}{|l|}{\head{A. Header, paragraf dan list}} & \multicolumn{1}{c|}{} \\	
	\hline
  	\rownumber & Nomor Bab ditulis dengan huruf Romawi Besar. & $\surd$ \\ 
  	\hline
  	\rownumber & Judul Bab ditulis seluruhnya dengan huruf besar, diketik tebal dengan ukuran 14pt, dan diatur supaya simetris, tanpa diakhiri dengan titik. & $\surd$ \\
  	\hline
  	\rownumber & Judul Sub Bab dicetak tebal tanpa diakhiri dengan titik. Semua kata diawali dengan huruf besar, kecuali kata penghubung dan kata depan. Kalimat pertama sesudah judul sub bab dimulai dengan alinea baru. Judul sub bab bila lebih dari satu baris maka ditulis satu spasi. & $\surd$ \\
	\hline
	\rownumber & Judul Sub Sub Bab diketik mulai dari batas tepi kiri dan dicetak tebal, hanya kata pertama diawali huruf besar, tanpa diakhiri dengan titik. Kalimat pertama sesudah judul sub sub bab dimulai dengan alinea baru. & $\surd$ \\
	\hline
	\rownumber & Nomor Sub Bab ditulis dengan angka Arab sesuai dengan nomor Bab diikuti dengan nomor urut Sub Bab. & $\surd$ \\
	\hline
	\rownumber & Nomor Anak Sub Bab ditulis dengan angka Arab sesuai dengan nomor Sub Bab diikuti dengan nomor urut Anak Sub Bab. & $\surd$ \\
	\hline
	\rownumber & Apabila terdapat bagian lebih lanjut dari Anak Sub Bab, judul diketik tanpa nomor dan menggunakan huruf tebal (bold). & $\surd$ \\
	\hline
	\rownumber & Alinea baru dimulai pada ketikan ke-6 dari batas tepi kiri ketikan. & $\surd$ \\
	\hline
	\rownumber & Semua item dalam suatu list (baik bernomor maupun yang menggunakan bullet) diakhiri dengan titik. & $\surd$ \\
	\hline
	\multicolumn{2}{|l|}{\head{B. Pengacuan pustaka di teks}} & \multicolumn{1}{c|}{} \\	
	\hline
	\rownumber & Menggunakan bentuk (nama, tahun) untuk mengacu pustaka dan nama (tahun) untuk mengacu ke penulis. Satu atau dua penulis: tuliskan nama akhir semua penulis. Tiga atau lebih penulis: tuliskan nama akhir penulis pertama diikuti dengan "et al." atau "dkk." & $\surd$ \\
	\hline
	\multicolumn{2}{|l|}{\head{C. Tabel dan gambar}} & \multicolumn{1}{c|}{} \\	
	\hline	
\rownumber & Tabel/Gambar sudah diberi judul (tabel di atas, gambar di bawah) dan nomor. Nomor berbentuk
X.Y, di mana X adalah nomor bab dan Y adalah nomor tabel/gambar pada bab tersebut. & $\surd$ \\
	\hline	
	\rownumber & Tabel/Gambar sudah diacu dan dijelaskan dalam teks. Pengacuan dengan menggunakan nomor (contoh: "Gambar 3.2"), bukan letak (contoh: "Gambar di bawah ini") & $\surd$ \\
	\hline
	\rownumber & Tabel/Gambar yang diambil dari pustaka lain sudah disertai dengan acuan pustaka. & $\surd$ \\
	\hline
	\multicolumn{2}{|l|}{\head{D. Daftar pustaka}} & \multicolumn{1}{c|}{} \\	
  	\hline
	\rownumber & Urut alfabet berdasarkan nama akhir penulis pertama. & $\surd$ \\
	\hline
	\rownumber & Nama semua penulis tertulis, tidak menggunakan "et al." atau "dkk.". & $\surd$ \\
	\hline
	\rownumber & Semua nama depan dari semua penulis disingkat, contoh "Pulungan, R." & $\surd$ \\
	\hline
	\rownumber & Tahun penerbitan muncul setelah nama-nama penulis. & $\surd$ \\
	\hline
	\rownumber & Semua pustaka yang diacu di teks dicantumkan di daftar pustaka. & $\surd$ \\
	\hline
	\rownumber & Daftar pustaka tidak berisi pustaka yang tidak pernah diacu di teks. & $\surd$ \\
	\hline
	\rownumber & Pustaka tanpa penulis menggunakan nama penulis "Anonim". & $\surd$ \\
	\hline
	\rownumber & Kelengkapan halaman, volume dan issue untuk pustaka konferensi dan journal. & $\surd$ \\
	\hline
	\rownumber & Kelengkapan penerbit, kota penerbit untuk buku, thesis dan disertasi. & $\surd$ \\
	\hline
\end{tabular}


\vspace{.4cm}
\noindent
\begin{tabular}{p{10cm}p{10cm}}
						& \@city,\space\today \\
Mengetahui pembimbing,	& Yang membuat pernyataan \\ [1.5cm]
\underline{\@firstsupervisor}	& \underline{\@fullname\space\space\space\space\space\space\space\space\space\space\space\space\space\space\space\space\space\space} \\
\@firstsupervisornip			& NIM. \@idnum
\end{tabular}
%%%%%%%%%%%%%%%%%%%%%%%
